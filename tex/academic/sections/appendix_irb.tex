\section{Appendix A: Study~A IRB Protocol}

\subsection{Administrative Details}
\begin{description}[leftmargin=0pt]
  \item[Title] Energy Cost of Overnight Erasure: REM Expenditure vs. Daytime Informational Entropy.
  \item[Investigators] Principal Investigator (PI) with co-investigators from sleep medicine, neuroimaging, and biostatistics; certified sleep technologists.
  \item[Sites] University Sleep Laboratory and Metabolic Chamber Core.
  \item[Funding] To be specified per grant or contract.
\end{description}

\subsection{Summary}
We assess whether mentally novel days increase brain energy use during REM sleep and whether overnight entropy reduction relates to next-day stress relief.

\subsection{Aims and Hypotheses}
\begin{enumerate}[label=Aim~\arabic*., leftmargin=*]
  \item Test the relationship between stage-resolved REM energy expenditure ($E_{\text{REM}}$) and daytime neural entropy ($\uncertainty_{\text{day}}$). Hypothesis: higher $\uncertainty_{\text{day}}$ predicts higher $E_{\text{REM}}$.
  \item Determine whether overnight entropy reduction ($\Delta \uncertainty$) predicts lower cortisol area-under-curve (AUC) and increased HRV. Hypothesis: greater $\Delta \uncertainty$ improves next-day physiological markers.
\end{enumerate}
Exploratory analyses evaluate targeted forgetting of distractor traces.

\subsection{Design}
Within-subject, counterbalanced three-night protocol (high-entropy, low-entropy, control). Sample size $n=48$ based on mixed model power simulations.

\subsection{Eligibility}
Inclusion: ages 18--40, habitual sleep 6.5--9~hours, right-handed, English fluent. Exclusion: sleep disorders, psychoactive medications, uncontrolled metabolic conditions, pregnancy, metal implants (PET subset), heavy caffeine or alcohol use, shift work.

\subsection{Procedures}
\begin{itemize}
  \item \textbf{Screening}: consent, sleep/medical history, Epworth Sleepiness Scale, toxicology, pregnancy test, one-week actigraphy.
  \item \textbf{Day protocol}: eight-hour task battery manipulating novelty with EEG, pupillometry, and confidence probes; standardized meals, no caffeine.
  \item \textbf{Overnight}: polysomnography, whole-room indirect calorimetry, optional FDG-PET during REM for subset ($n=16$).
  \item \textbf{Morning}: HRV, salivary cortisol, memory tests for consolidation vs. distraction.
\end{itemize}

\subsection{Measures and Outcomes}
Primary outcomes: $E_{\text{REM}}$ (kJ/min) and $\Delta \uncertainty$ (difference between day entropy and morning resting entropy). Secondary: cortisol AUC, HRV (RMSSD), sleep architecture, task performance.

\subsection{Risks and Mitigation}
Sleep disruption, electrode irritation, and, for PET participants, low-dose radiation. Mitigation includes flexible scheduling, hypoallergenic materials, and radiological oversight.

\subsection{Benefits}
No direct clinical benefit; contribution to understanding sleep, learning, and metabolism.

\subsection{Confidentiality}
Coded identifiers, encrypted storage, limited access, data retention for five years post-publication.

\subsection{Compensation}
\$75 per night plus \$50 per day session, \$150 PET supplement, \$50 completion bonus.

\subsection{Statistics}
Linear mixed-effects models, mediation analyses, sensitivity covariates (chronotype, prior sleep, menstrual phase), preregistered contrasts, code shared on OSF.

\subsection{Withdrawal}
Participation is voluntary; partial compensation pro-rated. Alternatives include non-participation.

\subsection{Adverse Event Reporting}
Unanticipated problems reported within five business days; data safety monitoring board not required for minimal risk.

\subsection{Consent Template}
Lay summary emphasizing purpose, procedures, risks, benefits, confidentiality, compensation, voluntariness, and contacts, followed by signature fields for participant and researcher.

\section{Background}

\subsection{The Thermodynamics of Information and the Landauer Limit}
In 1961, Rolf Landauer articulated a bridge between computation and physics: the principle that erasing one bit of information incurs an irreducible energetic cost. The lower bound for erasure at temperature $T$ is
\begin{equation}
  E_{\min} = k_B T \ln 2.
\end{equation}
This result reframed memory and learning as physical processes. Any logically irreversible operation---erasing a memory cell, resetting a register---dissipates energy as heat. Developments in stochastic thermodynamics and non-equilibrium information theory have confirmed the bound across colloidal particles, molecular switches, and quantum systems.

For biological systems that store, process, and erase information, the same constraint applies. Learning, consolidation, forgetting, and synaptic pruning become thermodynamic processes bounded by \landauer{}, with experimentally measurable energetic signatures.

\subsection{Learning as Compression: The Information-Theoretic View}
Within Shannon information theory, learning reduces uncertainty by compressing environmental structure into predictive codes. Kolmogorov complexity frames understanding as discovering the shortest algorithm that reproduces observations. Theories such as Schmidhuber's compression progress and Friston's free-energy principle characterize intelligence as redundancy minimization and prediction-error reduction.

Compression alone is insufficient. Efficient learners must delete obsolete data, shedding representations that no longer contribute to accurate prediction. In humans, this deletion manifests as memory pruning, synaptic downscaling during sleep, and the emotional process of closure. Erasure is thermodynamically expensive and risky; energy is committed irreversibly once the system judges that the new model subsumes the old.

\subsection{Confidence and the Energetics of Deletion}
Computational erasure is deterministic; biological cognition must estimate confidence. The brain must trust that its compressed model adequately explains the environment before deleting redundant scaffolds. This introduces a coupling between epistemic certainty and metabolic efficiency.

Synaptic plasticity and pruning are energy-intensive, drawing on glial metabolism. REM sleep and slow-wave oscillations---associated with consolidation and forgetting---show surges in glucose consumption alongside entropy reduction. Chronic stress, trauma, and rumination correspond to failures of confident deletion: unresolved loops remain active, consuming metabolic resources without delivering predictive efficiency.

\subsection{Biological Computation and the Energy--Information Nexus}
Living systems behave as thermodynamic information engines that consume free energy to reduce uncertainty. Neurons are metabolically costly: the brain is roughly \SI{2}{\percent} of body mass yet consumes \SI{20}{\percent} of resting energy. ATP demand spikes during synaptic plasticity, and mitochondrial density tracks regions of high computation. Thus cognition is a metabolic act---a controlled energy burn to reconfigure informational states.

At the systemic level, energy budgets spanning metabolism, stress response, and immune regulation can be framed as global optimization for minimizing prediction error. Unresolved uncertainty leads to elevated reserves (fat storage, inflammation), analogous to redundant encoding in computation: retaining multiple copies until deletion is safe.

\subsection{Toward a Unified Thermodynamic Theory of Learning}
Viewing learning as thermodynamically constrained compression and deletion bridges disciplines:
\begin{itemize}
  \item \textbf{Information theory}: Learning extracts structure and reduces entropy.
  \item \textbf{Thermodynamics}: Erasure dissipates energy in accord with \landauer{}.
  \item \textbf{Neuroscience}: Synaptic and metabolic processes instantiate these costs.
  \item \textbf{Psychology}: Emotional resolution corresponds to confident deletion.
  \item \textbf{Physiology}: Energy storage and inflammation embody deferred computation.
\end{itemize}
The human organism's metabolic and cognitive dynamics are two facets of a single physical process: converting energetic uncertainty into informational certainty.

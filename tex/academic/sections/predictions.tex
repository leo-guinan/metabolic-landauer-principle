\section{Predictions and Testable Hypotheses}

We translate the \MLP{} into empirically testable predictions linking entropy, confidence, and energy.

\subsection{Prediction 1: Epistemic Confidence Reduces Metabolic Load}
\textbf{Hypothesis.} Higher epistemic confidence reduces basal metabolic expenditure by lowering $\uncertainty$ and the need for redundancy buffering.

\textbf{Operationalization.} Independent variable: confidence measured via psychometric scales. Dependent variables: resting metabolic rate, blood glucose variability, cortisol, C-reactive protein (CRP), and heart rate variability (HRV).

\textbf{Prediction.} High-confidence participants display lower resting metabolic rate for equivalent cognitive load, reduced chronic cortisol and inflammatory markers, and higher HRV.

\subsection{Prediction 2: Cognitive Compression Correlates with Glucose Efficiency}
\textbf{Hypothesis.} Learning efficiency, quantified as compression ratio between input entropy and stored representation, correlates with glucose utilization efficiency.

\textbf{Operationalization.} Independent variable: entropy reduction via EEG/MEG or fMRI. Outcomes: FDG-PET glucose consumption, lactate accumulation, learning curves.

\textbf{Prediction.} Efficient learners consume less glucose per bit of uncertainty reduced, approaching the \landauer{} limit.

\subsection{Prediction 3: Emotional Resolution Releases Measurable Energy}
\textbf{Hypothesis.} Emotional resolution triggers acute energetic shifts corresponding to confident deletion of high-entropy emotional loops.

\textbf{Operationalization.} Interventions such as exposure or mindfulness therapy. Measures include VO$_2$, heart rate, skin conductance, cortisol, adrenaline, serotonin, EEG entropy, and subjective insight.

\textbf{Prediction.} Resolution moments exhibit transient energy release and sustained reduction in metabolic arousal.

\subsection{Prediction 4: Sleep Energy Expenditure Tracks Informational Entropy}
\textbf{Hypothesis.} REM sleep energy expenditure scales with informational entropy accumulated during wakefulness.

\textbf{Operationalization.} Wakefulness entropy index versus REM-phase glucose metabolism using polysomnography and FDG-PET.

\textbf{Prediction.} High-novelty days produce proportionally higher REM energy consumption.

\subsection{Prediction 5: Chronic Stress Represents a Stalled Deletion State}
\textbf{Hypothesis.} Chronic stress corresponds to high $\uncertainty$ with suppressed $\confidence$, yielding continuous metabolic arousal without resolution.

\textbf{Operationalization.} Cortisol, HRV, amygdala activation, hippocampal plasticity, rumination indices.

\textbf{Prediction.} Chronic stress profiles match systems with high entropy and blocked erasure pathways.

\subsection{Prediction 6: Flow States Approach Landauer Efficiency}
\textbf{Hypothesis.} Flow states reflect near-optimal coupling between entropy reduction and energy consumption.

\textbf{Operationalization.} Skill-challenge balance tasks with EEG, metabolic, and HRV measurements plus subjective flow scales.

\textbf{Prediction.} Flow minimizes energy per bit of prediction error corrected and maximizes neural coherence.

\subsection{Prediction 7: Social Trust Lowers Collective Energy Expenditure}
\textbf{Hypothesis.} Trust functions as distributed confidence, reducing redundant verification and energetic cost.

\textbf{Operationalization.} Cooperative tasks with high- versus low-trust groups; measures include calorimetry, communication entropy, and performance efficiency.

\textbf{Prediction.} Trust-primed groups expend less energy per successful outcome and exhibit reduced redundancy.

\subsection{Summary Table}
\begin{table}[h]
  \centering
  \caption{Predicted markers across domains.}
  \label{tab:prediction_summary}
  \begin{tabular}{llll}
    \toprule
    Domain & Hypothesis & Markers & Expected Outcome \\
    \midrule
    Neurological & Efficient compression lowers glucose/bit & EEG entropy, FDG-PET & High learning efficiency $\to$ low energy cost \\
    Emotional & Resolution reduces stress energy & HRV, cortisol, EEG coherence & Energy release at closure \\
    Metabolic & Uncertainty correlates with fat storage & CRP, insulin, BMI & Chronic uncertainty $\to$ energy hoarding \\
    Sleep & REM energy $\propto$ daily entropy & Polysomnography, FDG-PET & Novelty days $\to$ high REM energy \\
    Social & Trust increases collective efficiency & Communication entropy, calorimetry & Low redundancy $\leftrightarrow$ high trust \\
    \bottomrule
  \end{tabular}
\end{table}

\subsection{Experimental Paradigm Design}
We propose an Energy--Information Coupling Protocol (EICP):
\begin{enumerate}[label=\arabic*.]
  \item Measure baseline metabolic, neural, and confidence indices.
  \item Induce uncertainty via controlled novelty or challenge.
  \item Facilitate compression through feedback to elevate $\confidence$.
  \item Track energetic release following deletion-ready confidence.
  \item Compare high- versus low-confidence cohorts.
\end{enumerate}

\subsection{Overarching Prediction}
Across neurons, bodies, and groups, energy-use efficiency is bounded by compression capacity and confidence in deletion. Systems that learn without deleting accumulate energetic debt; systems that compress and delete with confidence achieve energetic harmony.
